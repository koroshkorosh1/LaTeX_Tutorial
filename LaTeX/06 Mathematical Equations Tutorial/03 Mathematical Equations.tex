\documentclass[12pt]{article}  
            \usepackage{mathtools}  
               \begin{document} 
                
                      % Equ(1)
                     \begin{equation*}  
                          F(x) &= \int^a_b \frac{1}{3}x^{4x}
                     \end{equation*}
                    
                      % Equ(2)
                     \begin{gather*} 
	                   (a+b)=a^2+b^2+2ab \\ 
	                   (a-b)=a^2+b^2-2ab
	                 \end{gather*}
                         
                      % Equ(3) 
                      \begin{equation*}  
                        \binom{n}{k} = \frac{n!}{k!(n-k)!}
                      \end{equation*}
                         
                      % Equ(4)  
                       \begin{equation*}  
                         f(x)=\frac{P(x)}{Q(x)} \ \ \textrm{and} 
	                     f(x)=\textstyle\frac{P(x)}{Q(x)}               
                       \end{equation*}
                         
                      % Equ(5)
                       \begin{equation*}  
                            \frac{1+\frac{a}{b}}{1+\frac{1}{1+\frac{1}{a}}} 
                       \end{equation*}
                            
                      % Equ(6)   
                       \begin{align*}  
                           \sin A \cos B &= \frac{1}{2}\left[ \sin(A-B)+\sin(A+B) \right] \\
                           \sin A \sin B &= \frac{1}{2}\left[ \sin(A-B)-\cos(A+B) \right] \\
                           \cos A \cos B &= \frac{1}{2}\left[ \cos(A-B)+\cos(A+B) \right] \\  
                       \end{align*} 
                            
                      % Equ(7)    
                       \begin{align*}  
                               {\frac {d}{dx}}\arctan(\sin({x}^{2}))=-2\,{\frac {\cos({x}^{2})x}{-2+
                                \left (\cos({x}^{2})\right )^{2}}} 
                       \end{align*} 
                            
                            
                      % Equ(8)     
                       \begin{align*}  
                                \frac{d}{dx}\left( \int_{0}^{x} f(u)\,du\right)=f(x)  
                       \end{align*} 
                             
                           
                         
                      % Equ(9)    
                       \begin{align*}  
                                 \displaystyle\sum\limits_{i=0}^n i^3
                       \end{align*} 
                              
                              
                      % Equ(10)       
                       \begin{pmatrix*}  
                                 1 & 2 & 3 \\
                                 0 & 1 & 4 \\
                                 4 & 2 & -1 
                       \end{pmatrix*} 
                              
                      % Equ(11)              
                       \begin{align*}  
                                 \displaystyle{\int 1 dx}	 
                       \end{align*}  
                         
                         
               \end{document}  