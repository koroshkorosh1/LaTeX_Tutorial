\documentclass[12pt]{article}   
\usepackage{mathtools}  
\usepackage{xfrac}  
            \begin{document}  
                \[ f(x) =  
                    \begin{cases}  
                     x^7 + 7x       & \quad \text{if } x \text{ is greater than 0}\\ % the text command is just used for the formatting  
                      0  & \quad \text{if } x \text{ is less than 0} % the \quad command maintains the distance between the text and the math variable  
                    \end{cases}  
               \]
                \vspace{5 em}
               \[
                   \begin{matrix}  
                         \begin{array}{c|c}  
                             Korosh & Mahdi \\   
                             \hline  
                                1 & 2  
                           \end{array}  
                    \end{matrix}  
               \]
               \vspace{5 em}
               \[  
                   \begin{pmatrix} 
                         a_{11} & a_{12} & \cdots & b_{1 n} \\  
                         a_{21} & a_{22} & \cdots & a_{2 n} \\ 
                         \vdots & \vdots & \ddots & \vdots \\  
                         a_{31} & a_{32} & \cdots & a_{3 n}   
                   \end{pmatrix}  
               \] 
               \vspace{5 em}
               \[  
                   \begin{bmatrix}  
                          A & B & C \\  
                          D & E & F \\  
                          G & H & I   
                   \end{bmatrix}  
                    =  
                   \begin{bmatrix*}  
                          J & K & L \\  
                          M & N & O \\  
                           P & Q & R  
                    \end{bmatrix*}  
               \]  
  
        \end{document}  